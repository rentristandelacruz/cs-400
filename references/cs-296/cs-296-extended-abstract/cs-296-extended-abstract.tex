\documentclass{article}
\usepackage{amssymb}
\usepackage{amsmath}
\usepackage{geometry}
\usepackage{graphicx}
\usepackage{float}

\geometry
{
a4paper,
bottom=20mm,
right=20mm,
left=20mm,
top=10mm,
}

\newcommand{\s}{\text{ }}

\title
{
Reformulation, Extension, and Application of the\\
Formal Framework for P Systems ({Extended Abstract})
}
\date{\today}
\author{Ren Tristan A. de la Cruz}

\begin{document}
\maketitle

% ================================================================================================ %

\begin{abstract}
\end{abstract}

\textit{Membrane computing} is a field of theoretical computer science that studies different models 
of computation known as \textit{P systems}. The term `\textit{P systems}' refers to a family of        
models of computation which are inspired by biological processes. P system models use abstractions     
of biological processes as computational operations. For example, different types of rules             
(operations) used by most P system variants are abstractions of processes like \textit{chemical        
reaction} and \textit{ion transport} that occur inside biological cells. Most P system variants use  
\textit{object symbols} as the objects of computation. One can think of these object symbols as        
abstraction of physical \textit{molecules} or \textit{ions}. P systems store \textit{multisets} of     
these object symbols inside regions enclosed by \textit{membranes}. A P system has a collection of     
these membranes with multisets of objects symbols inside. The membranes can be `connected' to each     
other to form a \textit{membrane structure}.                                                           
                                                                                                       
Aside from studying P systems themselves, membrane computing also deals with applications of           
P systems. P systems have been used to solve `hard' (NP-complete) problems. They have been used to     
model natural phenomena like biological oscillations, population dynamics, and sodium-potassium pump 
mechanism. Other P system applications include boolean circuit simulation, fault diagnosis models      
for electric locomotive, linguistic applications, etc.                                                 
                                                                                                       
There are tens, if not hundreds, of P system variants. Their syntax are well-defined but the           
semantics are often described in an informal manner. \emph{Formal framework} is an attempt to          
formalize not only syntax but also the procedural semantics of a wide variety of P systems. There      
are currently three versions of the framework, one for P systems with static membrane structures,      
one for P systems with dynamic membrane structures, and another for static P systems with              
input-output. The formal frameworks can be used to analyze, compared, and extended P systems.          
                                                                                                       
The \emph{first formal framework (FF1)} \cite{ff-static} is an attempt to formally define procedural 
semantics for a large number of well-known variants of tissue P systems with static membrane         
structures. The \emph{second formal framework (FF2)} \cite{ff-dynamic} is a similar attempt for P    
systems with dynamic membrane structures. The \emph{third formal framework (FF3)} \cite{ff-snp} is   
an extension of FF1 with the purpose of expressing spiking neural P systems \cite{snp} in this       
extended framework.                                                                                  
                                                                                                     
The formal frameworks use and formalize notions/concepts common to most of their target P systems.   
The main notions available in all three versions of the framework are the following:                 
                                                                                                     
\begin{enumerate}                                                                                    
\item \emph{configuration} of the system                                                             
\item \emph{rules} of the system                                                                     
\item \emph{applicability} of a combination of rules                                                 
\item \emph{allowed} combinations of rules                                                           
\item how to \emph{apply} a combination of rules                                                     
\item \emph{halting} condition for the system                                                        
\end{enumerate}      


In \cite{ff-using}, both FF1 and FF2 were used to implement different features of P system variants. 
These features include \emph{membrane thickness/polarity/labels}, \emph{rule priorities}, and        
\emph{membrane dissolution}. P system variants like \emph{catalytic P systems}, \emph{P systems with 
symport/antiport} \cite{sym-anti-port}, \emph{P colonies}, and \emph{probabilistic P systems} were   
also implemented as specific FF models. \cite{ff-using} shows that the formal framework can be used  
to model a wide variety of P systems with diverse features and semantics.                            
                                                                                                     
The FF can be used to compare different P system variants. By translating P system variants to their 
corresponding FF models, one can directly compare these FF models. The FF can be used a common       
language to analyze, compare and extended different P system variants. 

This research proposal has three aspects: (1) the reformulation of the formal framework, (2) the      
extension of the formal framework, and (3) the application of the formal framework. 

                                                                                                     
Reformulating the formal framework means changing the framework by changing the notions/concepts     
used or using different formalizations for these notions but not affecting the usefulness of the     
framework. Extending the framework means adding new notions and formalizations to extended the       
scope or usefulness of the framework. An extended framework can mean it can model more P system      
variants or that there are more notions in the framework that can provide more insights to the       
workings of existing `supported' P system variants. Application of the framework means using the     
framework to analyze, compare, and/or extended existing P system variants. 

The following are the specific objectives of this proposal:                                          
                                                                                                     
\begin{enumerate}                                                                                    
   \item (Reformulation) Combine FF2 and FF3 into a single formal framework. This involves the       
         addition of the $Input$ and $Output$ functions from FF3 to FF2. It also involves the use of 
         regular multiset languages for \emph{permitting} or \emph{forbidding} conditions of the     
         interaction rule. The purpose of this objective is to have a single formal framework (FF)   
         that can be used of static or dynamic P systems.                                            
   \item (Reformulation) Reformulate the interaction rule in FF (from objective 1) in a              
         \emph{bottom-up} manner instead of the \emph{top-down} approach of the FF. The rule in the  
         FF (or specifically FF2) contains 11 components because it is trying to be the most general 
         and unrestricted version of a rule such that the rule types from the P system variants are  
         simply restricted versions of the more general FF interaction rule. We call this approach   
         of finding the most general and unrestricted form of the rule as \emph{top-down}. A rule    
         can instead be defined as a `combination' of simpler `elementary' rules. We start from the  
         \emph{bottom} with this `elementary' rules and use them to define a general rule which is   
         a combination of these `elementary' rules.                                                  
   \item (Application) Perform a comprehensive survey of the different P system variants and use the 
         FF to create the equivalent FF models of the different P system variants.                   
   \item (Extension) While doing the comprehensive survey of P system variants, if there are         
         variants that are difficult or impossible to create an FF model for, formalize the features 
         of these variants and use them to extended the FF. Examples for such P system variants with 
         features that are not (directly) represented are available in \cite{polymorphic} and        
         \cite{rule-create}.                                                                         
   \item (Application) Create a simulator for FF models. Combining this simulator with the FF models 
        of the P systems from the survey (objective 4) will result in a fairly general simulator     
        than can simulate a wide variety of P systems.                                               
\end{enumerate} 

\bibliographystyle{plain}
\bibliography{cs-296-extended-abstract}

% ================================================================================================ %
\end{document}
